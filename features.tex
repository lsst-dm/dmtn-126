\section{Features and User Stories}
\label{sec:features}

As a <type of user>, I want <some feature> so that <some reason>. 

\subsection{Full Focal Plane Visualization}

Commissioning, Camera, and DRP expressed interest in being able to load a full focal plane and zoom continuously
from a large scale view down to pixel-level images.  In order to make pixel-level information useful, any
viewer will also need to support overplotting of mask planes.

Tony Johnson's image viewer, developed for the camera team, currently supports raft-level images and could be
extended to support full focal planes.  Luis da Costa has also developed a large scale viewer to handle preview
data as it comes off the camera.  This viewer can handle full focal planes.  There is talk of combining the
development effort on these two tools.

There are third party packages that can do large scale image
visualization, but these involve resampling images onto some dynamic pixellization of the sky (i.e.HEALPIX) and
thus, on some level, erase the information contained in individual LSST pixels.  The principal barrier to just
adopting Tony Johnson's image viewer is that it is enabled my a massive hardware investment, which may make it
prohibitive for hundreds of users to be viewing hundreds of focal planes simultaneously.

\subsection{Compare Images}
\begin{itemize}
\item{blinking}
\item{side-by-side}
\item{Locking on WCS or pixels coordinates}
\item{Locking on scale and limits}
\end{itemize}

All groups polled requested the ability to directly compare two images, by locking WCS or pixel coordinates, scale and limits, and both blinking and looking side by side.
After locking users want to be able to pan and zoom while viewing side by side or blinking.
Two groups also requested a ``crossfade'' option as ``nice to have'' rather than essential.
Crossfade means that two images are overlaid at the same time, and the user moves a slider to shift the relative weighting of each image.

Using ds9 on lsst-dev this is possible and easy (for frequent users of ds9).
Using Firefly or matplotlib on Nublado, it may be possible, but it is not easy.

User stores
\begin{itemize}
\item{As a pipelines developer,  I want to be able to blink two locked images so that I can assess the effect of an algorithm modification.}
\item{As a camera team developer,  I want to be able to change the scale limits of two side-by-side images by hand, so that I can see the overscan.}
\end{itemize}

