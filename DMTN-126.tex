\documentclass[DM,authoryear,toc]{lsstdoc}
% lsstdoc documentation: https://lsst-texmf.lsst.io/lsstdoc.html

\input{meta}

% To add a short-form title:
% \title[Short title]{Title}
\title{Image Display WG}

\author{%
Yusra AlSayyad (chair),
Scott Daniel,
Gregory Dubois-Felsmann,
Simon Krughoff,
Lauren MacArthur,
John Swinbank
}

\setDocRef{DMTN-126}
\setDocUpstreamLocation{\url{https://github.com/lsst-dm/dmtn-126}}

\date{\vcsDate}

\setDocAbstract{%
This document describes the findings of the LSST DM Image Display Working Group.
}

% Change history defined here.
% Order: oldest first.
% Fields: VERSION, DATE, DESCRIPTION, OWNER NAME.
% See LPM-51 for version number policy.
\setDocChangeRecord{%
  \addtohist{1}{YYYY-MM-DD}{Unreleased.}{Yusra AlSayyad}
}

\begin{document}

\maketitle

\section{Introduction}

The QA Working Group (QAWG) was formed in mid-calendar-2017, with a wide-ranging remit to suggest improvements to processes and tools used for quality assurance across the Data Management subsystem, following the charge defined in \citeds{LDM-622}.
The QAWG produced an extensive report (\citeds{DMTN-085}), which touched on issues in the area of image display and visualization.
However, due to uncertainty around project scope and the future of development of image display tools within DM, the QAWG did not issue effective recommendations in this area.

The Image Display Working Group (IDWG) was constituted in summer 2019 with the primary goal of rectifying this shortcoming, as well as issuing recommendations on image display for the purposes of commissioning (which fell outside the QAWG scope) and considering whether useful recommendations can be made about tool development in support of diagnostic work within the Camera Subsystem.
The detailed terms of the Working Group are provided in \citeds{LDM-702}.

\section{Approach}

The IDWG identified three ways to approach its charge.

\begin{enumerate}

  \item{
    We considered the various contexts or environments in which a scientist or developer might wish to view images, and established how well these are served by current tools.
    For example, these environments include a developer working on a standalone laptop, or a scientist using the LSST Science Platform.
    This analysis is presented in \S\ref{sec:env}.
  }

  \item{
    We surveyed groups of key stakeholders across the LSST Project, including DM pipeline developers, members of the Commissioning Team, and members of the Camera Team, to understand what they regard as the most important use cases for visualization, and how well those use cases are served by existing tools.
    This analysis is presented in \S\ref{sec:features}.
  }

  \item{
    We identified existing tools --- including those developed within DM, those produced by other subsystems within LSST, and those available in the wider community --- to assess their capabilities and relevance to the key use cases which have been identified.
    This analysis is presented in \S\ref{sec:tools}.
  }

\end{enumerate}

Finally, in \S\ref{sec:conc} we present conclusions and recommendations drawn from the above.

\section{Environments}
\label{sec:env}

There are eight environments in which a DM member would view images.

Simon's table.

Simon's chart in words.

We focused on the lsst-dev and Nublado environments, which are partially implemented.

DM requested these essential features.
Features = some action users want to be able to complete or some characteristic of the image viewer.
Use cases are the commissioning,  science validation, pipelines writing tasks that prompt these actions.

Some of these features are currently possible and easy on lsst-dev and Nublado; some are not.

We distinguish between ``possible'' and ``easy.''

\section{Features and User Stories}
\label{sec:features}

As a <type of user>, I want <some feature> so that <some reason>. 

\subsection{Full Focal Plane Visualization}

Commissioning, Camera, and DRP expressed interest in being able to load a full focal plane and zoom continuously
from a large scale view down to pixel-level images.  In order to make pixel-level information useful, any
viewer will also need to support overplotting of mask planes.

Tony Johnson's image viewer, developed for the camera team, currently supports raft-level images and could be
extended to support full focal planes.  Luis da Costa has also developed a large scale viewer to handle preview
data as it comes off the camera.  This viewer can handle full focal planes.  There is talk of combining the
development effort on these two tools.

There are third party packages that can do large scale image
visualization, but these involve resampling images onto some dynamic pixellization of the sky (i.e.HEALPIX) and
thus, on some level, erase the information contained in individual LSST pixels.  The principal barrier to just
adopting Tony Johnson's image viewer is that it is enabled my a massive hardware investment, which may make it
prohibitive for hundreds of users to be viewing hundreds of focal planes simultaneously.

\subsection{Compare Images}
\begin{itemize}
\item{blinking}
\item{side-by-side}
\item{Locking on WCS or pixels coordinates}
\item{Locking on scale and limits}
\end{itemize}

All groups polled requested the ability to directly compare two images, by locking WCS or pixel coordinates, scale and limits, and both blinking and looking side by side.
After locking users want to be able to pan and zoom while viewing side by side or blinking.
Two groups also requested a ``crossfade'' option as ``nice to have'' rather than essential.
Crossfade means that two images are overlaid at the same time, and the user moves a slider to shift the relative weighting of each image.

Using ds9 on lsst-dev this is possible and easy (for frequent users of ds9).
Using Firefly or matplotlib on Nublado, it may be possible, but it is not easy.

User stores
\begin{itemize}
\item{As a pipelines developer,  I want to be able to blink two locked images so that I can assess the effect of an algorithm modification.}
\item{As a camera team developer,  I want to be able to change the scale limits of two side-by-side images by hand, so that I can see the overscan.}
\end{itemize}



\section{Existing Tools}
\label{sec:tools}

\section{Conclusions}
\label{sec:conc}

\appendix

\section{References}
\label{sec:bib}
\bibliography{lsst,lsst-dm,refs_ads,refs,books}

\section{Abbreviations and Definitions}
\label{sec:acronyms}
\input{acronyms.tex}

\end{document}
