\documentclass[DM,authoryear,toc]{lsstdoc}
% lsstdoc documentation: https://lsst-texmf.lsst.io/lsstdoc.html

\newcommand{\yusra}[1]{{\it \textcolor{green}{[YA: #1]}}}

\input{meta}

% To add a short-form title:
% \title[Short title]{Title}
\title{Image Display WG}

\author{%
Yusra AlSayyad (chair),
Scott Daniel,
Gregory Dubois-Felsmann,
Simon Krughoff,
Lauren A. MacArthur,
John Swinbank,
Chris Waters
}

\setDocRef{DMTN-126}
\setDocUpstreamLocation{\url{https://github.com/lsst-dm/dmtn-126}}

\date{\vcsDate}

\setDocAbstract{%
This document describes the findings of the LSST DM Image Display Working Group.
}

% Change history defined here.
% Order: oldest first.
% Fields: VERSION, DATE, DESCRIPTION, OWNER NAME.
% See LPM-51 for version number policy.
\setDocChangeRecord{%
  \addtohist{1}{YYYY-MM-DD}{Unreleased.}{Yusra AlSayyad}
}


% Recommendation macro copied from dmtn-085:


% Hyperref plays havoc with my crazy recommendation linking TOCs if we let it
% turn the section name into a hyperlink, so we use the page numbers instead.
\hypersetup{linktoc=page}

\makeatletter
\newcommand{\printrecs}{%
  \section{Recommendations}%
  \label{sec:recs}
  \begin{enumerate}[leftmargin=7em,label=QAWG-REC-\arabic*:]%
  \def\@noitemerr{\@latex@warning{Empty objective list}}%
  \@starttoc{rec}%
  \end{enumerate}%
}
\newcommand{\l@rec}[2]{#1}
\newenvironment{recbox}
{
  \begin{center}
  \begin{minipage}[h]{.85\linewidth}
    \begin{tcolorbox}[colback=green!5!white,colframe=green!75!black,title=\textbf{Recommendation}]
}
{
    \end{tcolorbox}
  \end{minipage}
  \end{center}
}
%
% \begin{recommendation}{rec:label}{Brief summary}
% Explanatory text, if any.
% \end{recommendation}
\NewEnviron{recommendation}[2]
 {%
  \label{#1}%
  \addcontentsline{rec}{rec}{%
    \noexpand\unexpanded{\unexpanded\expandafter{\item{#2 (\S\ref{#1})}}}%
  }%
  \begin{recbox}
  \emph{#2.} \par \BODY%
  \end{recbox}
 }%


\begin{document}

\maketitle

\section{Introduction}

The QA Working Group (QAWG) was formed in mid-calendar-2017, with a wide-ranging remit to suggest improvements to processes and tools used for quality assurance across the Data Management subsystem, following the charge defined in \citeds{LDM-622}.
The QAWG produced an extensive report (\citeds{DMTN-085}), which touched on issues in the area of image display and visualization.
However, due to uncertainty around project scope and the future of development of image display tools within DM, the QAWG did not issue effective recommendations in this area.

The Image Display Working Group (IDWG) was constituted in summer 2019 with the primary goal of rectifying this shortcoming, as well as issuing recommendations on image display for the purposes of commissioning (which fell outside the QAWG scope) and considering whether useful recommendations can be made about tool development in support of diagnostic work within the Camera Subsystem.
The detailed terms of the Working Group are provided in \citeds{LDM-702}.

\section{Approach}

The IDWG identified three ways to approach its charge.

\begin{enumerate}

  \item{
    We considered the various contexts or environments in which a scientist or developer might wish to view images, and established how well these are served by current tools.
    For example, these environments include a developer working on a standalone laptop, or a scientist using the LSST Science Platform.
    This analysis is presented in \S\ref{sec:env}.
  }

  \item{
    We surveyed groups of key stakeholders across the LSST Project, including DM pipeline developers, members of the Commissioning Team, and members of the Camera Team, to understand what they regard as the most important use cases for visualization, and how well those use cases are served by existing tools.
    This analysis is presented in \S\ref{sec:features}.
  }

  \item{
    We identified existing tools --- including those developed within DM, those produced by other subsystems within LSST, and those available in the wider community --- to assess their capabilities and relevance to the key use cases which have been identified.
    This analysis is presented in \S\ref{sec:tools}.
  }

\end{enumerate}

Finally, in \S\ref{sec:conc} we present conclusions and recommendations drawn from the above.

\section{Environments}
\label{sec:env}

There are different environments in which a DM use would view images.
Different tools are appropriate for different environments.

\begin{itemize}
\item{\textbf{Laptop:} Have complete freedom of tools, for small amounts of data copied over}
\item{\textbf{Project dev server (\texttt{lsst-dev})}: Freedom of tools with image data disks mounted and available. X Windows forwarding.}
\item{\textbf{Science Platform:}  No X windows, need tools for JupyterLab and web.}
\end{itemize}

Therefore it is necessary to specify the environments associated with each recommendation.
We focused recommendations  on the lsst-dev and especially the Science Platform environments, for which the image display features are only  partially implemented.

DM requested the following  essential features.
Features are some action users want to be able to complete or some characteristic of the image viewer.
Use cases are the commissioning,  science validation, pipelines writing tasks that prompt these actions.
Some of these features are currently possible and easy on lsst-dev and Nublado; some are not.
We distinguish between ``possible'' and ``easy.''

\section{Features and User Stories}
\label{sec:features}
\section{Features and User Stories}
\label{sec:features}

As a <type of user>, I want <some feature> so that <some reason>. 

\subsection{Full Focal Plane Visualization}

Commissioning, Camera, and DRP expressed interest in being able to load a full focal plane and zoom continuously
from a large scale view down to pixel-level images.  In order to make pixel-level information useful, any
viewer will also need to support overplotting of mask planes.

Tony Johnson's image viewer, developed for the camera team, currently supports raft-level images and could be
extended to support full focal planes.  Luis da Costa has also developed a large scale viewer to handle preview
data as it comes off the camera.  This viewer can handle full focal planes.  There is talk of combining the
development effort on these two tools.

There are third party packages that can do large scale image
visualization, but these involve resampling images onto some dynamic pixellization of the sky (i.e.HEALPIX) and
thus, on some level, erase the information contained in individual LSST pixels.  The principal barrier to just
adopting Tony Johnson's image viewer is that it is enabled my a massive hardware investment, which may make it
prohibitive for hundreds of users to be viewing hundreds of focal planes simultaneously.

\subsection{Compare Images}
\begin{itemize}
\item{blinking}
\item{side-by-side}
\item{Locking on WCS or pixels coordinates}
\item{Locking on scale and limits}
\end{itemize}

All groups polled requested the ability to directly compare two images, by locking WCS or pixel coordinates, scale and limits, and both blinking and looking side by side.
After locking users want to be able to pan and zoom while viewing side by side or blinking.
Two groups also requested a ``crossfade'' option as ``nice to have'' rather than essential.
Crossfade means that two images are overlaid at the same time, and the user moves a slider to shift the relative weighting of each image.

Using ds9 on lsst-dev this is possible and easy (for frequent users of ds9).
Using Firefly or matplotlib on Nublado, it may be possible, but it is not easy.

User stores
\begin{itemize}
\item{As a pipelines developer,  I want to be able to blink two locked images so that I can assess the effect of an algorithm modification.}
\item{As a camera team developer,  I want to be able to change the scale limits of two side-by-side images by hand, so that I can see the overscan.}
\end{itemize}



\section{Existing Tools}
\label{sec:tools}
\yusra{Can I imagine writing an \texttt{afwDisplay} for this tool? Do they fulfill any use cases}

\subsection{Firefly}
\yusra{TO DO: Gregory}

Gregory says on Slack:
It's easy to imagine adding a wide variety of new features to Firefly at relatively modest cost, especially, as I mentioned in the meeting last week, ones that mainly involve writing more `afw`-oriented Python interfaces to existing features.
The sticking point we have, and what was behind my question about acceptability to the pipeline developer community last week, is that there are some performance-related issues that are inherent to the client-server architecture, and that would require significant engineering to address (e.g., by moving more of the data and computation to the client, or by doing more cacheing of likely-to-be-requested data). That engineering is not inconceivably difficult - Trey thinks about this kind of stuff all the time - but it may be a disproportionate amount of work compared to the perceived near-term benefit. (edited)
In some cases this is behind the difficulty in addressing what may at first sight appear to be simple UI issues. Merlin asked me, for instance, if we could connect the mouse scroll wheel UI events to the zoom-in and zoom-out actions. Just doing that is (I assume - I'm not a hard-core Javascripter yet) nearly trivial, but it would probably not yield a good UX, because the system is not designed to handle the high rates of actions that would produce.
So Merlin's take on the situation is that very fluid interactive response to zooming - and to rescaling - are more important to him than ``features'' like the mask-plane overlay controls (which he still really likes), and that as a result he's likely to prefer staying with DS9 in preference to starting to use Firefly in his daily work.

(To re-emphasize another earlier point: this is, and should be, totally OK, a choice up to individual developers, and a key motivation to have a ``neutral'' API like `afw.display` in the system. We're not trying to mandate tool use, but to decide where we can get the greatest net benefit to the user community from any additional LSST-funded tool development.) (edited)

We have been gradually making changes to improve the responsiveness of the UI by moving work from the server to the client - for instance, rotations are now done client-side and are really fast - but this is ``big'' work that is probably not realistically doable within the post-descope constraints - and also is not driven strongly by Firefly's other IPAC-side applications. (edited)
From the WG point of view, then, I think we need to understand how this ``fluid response? issue rates in the priorities of the user communities we have. That will then help us assess where best to invest resources. (edited)
(end of essay)


\subsection{Ginga (+Astrowidgets}
\yusra{TO DO: Chris or Yusra}
Things Chris learned at Python in Astronomy, plus a day's worth of experimenting.
Have not yet got \texttt{display\_ginga} working.
Chris has written a quick (~1 hour effort) GUI for interacting with the Ginga display in a notebook; looks pretty slick!
Can display mask planes, by adopting code from (but not integrating with) \texttt{display\_ginga}.
Add regions by clicking on the image; get the results through Python callbacks.
Understands FITS headers; can access them.
Which can actually be populated by metadata provided by the code; it doesn't have to be header.
Not clear what the semantics of this should be.
One could imagine extending afwDisplay with some generic notion of image metadata.
There are handy keybindings.
AstroWidgets acts largely as a wrapper around Ginga.
Chris is working on a way of panning through multiple frames.
Continuing development of AstroWidgets; they claim to now support all of the DS9 region types.

\subsection{JS9}

\yusra{TO DO: Simon}
Eric Mandel actively developing.
Hard to do mask overlays, due to fundamental differences between how DS9 and JS9 handle multiple images in the same app.
Not clear what the multi-client environment is like ? one monolithic server with many clients, or each client spawning its own server?
But note that there may be constraints on this because the browser and the python interpreter would need to have access to the same environment.
As far as we understand at the moment, matching the sort of mask overlay that we currently have in Firefly/DS9 isn't possible.
But that may just be a matter of not fully understanding JS9's capabilities.
The pyjs9 client library wraps a RESTful API exposed by the server.


\subsection{LSST Camera Image Viewer}
\label{sec:existing_tools:camera}

The LSST Camera Team has developed a bespoke image viewing tool to support their activities.
As such, it is primarily aimed at examining raw pixel data recorded from the camera to diagnose hardware issues.
This viewer is being developed by Tony Johnson (SLAC); much of the material below is adapted from his comments.
Given the rapid pace of ongoing development, some of the discussion below may soon (or already) be outdated.

The current version of the Camera Image Viewer is available on the web at \url{https://lsst-camera-dev.slac.stanford.edu/FITSInfo/}.

\subsubsection{Capabilities}

Given its audience and goals, it does not provide facilities which are require for astronomical analysis of the data.
For example, the viewer has no concept of world coordinate system: it simply operates in focal plane pixel coordinates.

Instead, the viewer is designed to provide high-performance interactive display of images corresponding to a full focal plane or subsections thereof.
The development plan prioritizes:

\begin{itemize}

  \item{A web-based interface for selecting and accessing data;}
  \item{Scaling from full focal-plane to individual amplifiers, with real-time panning and zooming;}
  \item{The ability to select, display and analyze regions by hardware device (e.g. particular CCDs, amplifiers, etc.);}
  \item{The ability to perform hardware diagnostic functions on the selected image region;}
  \item{The ability to trigger DM algorithms to run on the selected region;}
  \item{The ability to track and overlay the values of diagnostic measurements with time.}

\end{itemize}

At time of writing, not all of this functionality is currently available.

\subsubsection{Implementation details}

The tool consists of two separate pieces (the ``browser'' and the ``viewer'').
The former is a relatively simple web front end to the Camera image database which enables the user to select data of interest.
The latter is more technically interesting, as it must cope with the technical challenge of making available extremely large quantities of image data to the user in a fast, interactive way.

The viewer is based on the International Image Interoperability Framework\footnote{\url{https://iiif.io}} (IIIF), which defines a standardized API for making image data available on the web at scale.
A primary advantage of this is that existing open-source software is used as the basis of the system.
In particular, the Camera Image Viewer is based on:

\begin{itemize}

  \item{Cantaloupe\footnote{\url{https://cantaloupe-project.github.io}}, a server-side system for reading image data from storage and providing it to clients using IIIF standards;}
  \item{OpenSeadragon\footnote{\url{https://openseadragon.github.io}}, a browser-based Javascript library for receiving and displaying image data.}

\end{itemize}

Both of these tools provide for extensive customizability.
In particular, Cantaloupe has been extended to read data from the FITS files delivered by the camera, while OpenSeadragon is being extended to provide the various user-facing interaction and analysis facilities which are required.

It is worth noting that, although data is stored on the server in FITS format, it is converted to JPEG for transmission to the client.
It is therefore lossily compressed: the original pixel values are not directly available in the image stream.

Also note that --- regardless of the software system used --- loading and displaying full-focal-plane data at a low enough latency to provide a satisfactory interactive experience places a substantial load on the back-end storage system.
This will, of course, scale with the number of clients accessing the system, but one should expect to make a significant investment in hardware underlying the image viewer.

\subsubsection{Application to DM use cases}

The Camera Image Viewer provides an impressively high-performance interactive visualization of the full focal plane in a way that is not easily available through other tooling.
As such, it is of obvious value to the DM Subsystem.

However, the are a number of important limitations:

\begin{itemize}

  \item{The viewer does not have any concept of astronomical coordinate systems or of reprojecting data onto the celestial sphere;}
  \item{The level of analytic functionality currently available is limited, although more is in development;}
  \item{Integration with DM standard display interfaces (\texttt{afw.display}, etc) seems challenging;}
  \item{The current system is tightly coupled to the raw FITS images being produced by the camera: it would require substantial work to access processed data from a Butler repository.}

\end{itemize}

Deploying the system as it now stands at the Data Facility would require some investment in both hardware, to provide the low-latency data access required, and in development, to make it possible to retrieve and display data from Butler repositories rather than just camera raw FITS files.
However, making this investment would provide DM with a critical capability that is not easily available from elsewhere.


\subsection{HscMap}

\yusra{TO DO: Lauren}
Asked for feedback from Subaru/HSC to see if they have any inputs.
They sent some documentation videos!
http://hscmap.mtk.nao.ac.jp/hscMap4/app/help.html
http://hscmap.mtk.nao.ac.jp/hscMap4/jupyterlab-hscmap/docs/
https://hscmap.mtk.nao.ac.jp/hscMap4/app/
But no real user feedback.
The data is postprocessed before being fed to HscMap.
Which helps explain the speed.
``Full sky almost to pixel level''.
Lupton-sanctioned colours!
Can change colormap of pre-computed images, not on the original data.
``The algorithm is applied post-stretch''.
Interfaces with a database for catalog queries.
Can add your own images and CSV format catalogs.
Would be really useful to simply visualize the whole LSST sky.
Can be incorporated into JupyterLab.
But it's not clear how deep this integration goes.
Suspect there is no mask overlay support.
Chief selling-points are speed and colours, together with ability to overplot objects from a database.
We do not know what the backing format is; we know it used not to be HiPS, but that may have changed.

\subsection{Aladinlite}

Aladinlite is a tool for viewing images on entire-sky scales that allows users
to zoom down to smaller scales defined at the time the images were ingested.
It enables this functionality by storing images in the HiPS (Hierarchical
Progressive Survey) schema [citation to 2015A\&A...578A.114F].  In the HiPS
schema, images are sampled onto iteratively more refined healpixel grids.  These
resamplings are stored on a central server which allows users connecting to that
server to smoothly transition between levels of refinement, down to the
original, finest healpixel sampling.  While this seems to enable much of the
functionality we have recommended, specifically the ability to transition from
full focal plane inspection down to single amplifier inspection, the fact that
the images must be resampled onto a pixel grid that is defined in sky
coordinates means that the raw, sensor-level pixel information is ultimately
lost.  Aladinlite was designed to enable scientists to explore images at the
entire-sky level.  It was not meant to enable engineers to inspect pixel data
from sensors.  While Aladinlite will likely have a role to play in serving LSST
image data to the scientific community, it is unlikely to be helpful during the
process of commissioning the LSST hardware and software.

\subsection{Miscellaneous}
\subsubsection{ExpViewer}
\label{sec:existing_tools:misc:expviewer}

ExpViewer is being developed by a team led by Luiz da Costa in Brazil to provide rapid display of image data received from the camera in the operational era.
A preview version is currently available online at \url{http://expviewer.linea.gov.br}.

ExpViewer is built upon the same International Image Interoperability Framework as the Camera Image viewer (\S\ref{sec:existing_tools:camera}), and also makes use of OpenSeadragon for in-browser display.
However, rather than directly reading data from FITS stored on disk, images are converted to TIFF format when they are ingested by the tool.

ExpViewer and the Camera Image Viewer have a lot in common, but, at time of writing, the Camera Image Viewer provides a wider range of functionality, while ExpViewer provides a simpler and (arguably) more attractive user interface.
The WG understands that discussions are ongoing between the ExpViewer and Camera Image Viewer teams to identify commonalities, avoid duplication, and increase collaboration, but outcomes of this are currently unclear.

Given its ability to work with FITS data (albeit in the camera raw specific form) and its rapid pace of development, the WG suggests that the Camera Image Viewer provides a more attractive tool for potential adoption by DM.


\subsubsection{WorldWideTelescope}
\yusra{TO DO: yusra}
Peter Williams / WorldWideTelescope integrated HSC data with HSC.
Used a tiling scheme named TOAST; and a tool named toasty(?)
Took about 15 minutes to process a tract, but could likely be made faster with effort.
Addresses the whole sky visualization use case.


\yusra{TO DO: format feature vs. tool table (Yusra)}


\section{Recommendations}
\label{sec:conc}

% TO DO: John, Simon, Lauren, Gregory, Yusra: please review and edit
The recommendations are based on an assessment of the requested features, supporting use cases, and existing toolsets evaluated.
One currently unsupported feature is large image display.
Users across DM and commissioning (section \ref{sec:features:focal_plane}) will need to see a full-focal plane or full-tract image in its entirety and be able to zoom in to the individual pixels.
Of the existing tools evaluated, the LSST Camera Image Viewer was the most impressive in terms of responsiveness in the zooming.
The Camera Team reports that Tony Johnson is actively developing the viewer to include additional features.
Therefore adopting this will prevent duplication of effort.

However, the Camera Team reports that the viewer owes its responsiveness to a set of dedicated hardware.
Therefore, we recommend that DM
\begin{recommendation}{camera_viewer}{Provide a capability for fast display of full-focal-plane images for all data ingested at the LSST Data Facility based on the LSST Camera Image Viewer}
\end{recommendation}

Users across project are most proficient with DS9. Therefore, we recommend that DM

\begin{recommendation}{ds9}{Make ds9 available through commissioning and early operations by either:
\begin{itemize}
\item{Maintaining a development server (lsst-dev equivalent) for project staff,}
\item{Connecting LSP to local ds9 via SAMP or XPA or}
\item{Through WebDAV}
\end{itemize}}
The feasibility of these options should be evaluated
\end{recommendation}

Both JS9 and AstroWidgets show promise for the Jupyterlab environment.
JS9 is snappy, has the same interface as DS9, mouse gesture stretch, and on the fly smoothing.
Ginga/AstroWidgets is snappy, extensible by python programmers, and it is easy to run arbitrary python code on selected pixels.
Therefore we recommend that DM
\begin{recommendation}{js9_ginga}{Support at least one of JS9 / AstroWidgets + Ginga on Science Platform}
A bake-off is required to determine which to support
\end{recommendation}


Finally, firefly farther ahead feature-wise.
Firefly has a few unique features.
These include its intuitive mask plane toggle and that users can open in a window on another monitor (or across the country).
Therefore we recommend that DM
\begin{recommendation}{firefly}{Continue to provide support for the Firefly tool, including some level of resourcing for servicing emergent feature requests}
\end{recommendation}


\appendix

\section{References}
\label{sec:bib}
\bibliography{lsst,lsst-dm,refs_ads,refs,books}

\section{Abbreviations and Definitions}
\label{sec:acronyms}
\input{acronyms.tex}

\end{document}
