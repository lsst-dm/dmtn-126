\documentclass[DM,authoryear,toc]{lsstdoc}
% lsstdoc documentation: https://lsst-texmf.lsst.io/lsstdoc.html

\input{meta}

% To add a short-form title:
% \title[Short title]{Title}
\title{Image Display WG}

\author{%
Yusra AlSayyad (chair),
Scott Daniel,
Gregory Dubois-Felsmann,
Simon Krughoff,
Lauren MacArthur,
John Swinbank
}

\setDocRef{DMTN-126}
\setDocUpstreamLocation{\url{https://github.com/lsst-dm/dmtn-126}}

\date{\vcsDate}

\setDocAbstract{%
This document describes the findings of the LSST DM Image Display Working Group.
}

% Change history defined here.
% Order: oldest first.
% Fields: VERSION, DATE, DESCRIPTION, OWNER NAME.
% See LPM-51 for version number policy.
\setDocChangeRecord{%
  \addtohist{1}{YYYY-MM-DD}{Unreleased.}{Yusra AlSayyad}
}

\begin{document}

\maketitle

\section{Introduction}

The QA Working Group (QAWG) was formed in mid-calendar-2017, with a wide-ranging remit to suggest improvements to processes and tools used for quality assurance across the Data Management subsystem, following the charge defined in \citeds{LDM-622}.
The QAWG produced an extensive report (\citeds{DMTN-085}), which touched on issues in the area of image display and visualization.
However, due to uncertainty around project scope and the future of development of image display tools within DM, the QAWG did not issue effective recommendations in this area.

The Image Display Working Group was constituted in summer 2019 with the primary goal of rectifying this shortcoming, as well as issuing recommendations on image display for the purposes of commissioning (which fell outside the QAWG scope) and considering whether useful recommendations can be made about tool development in support of diagnostic work within the Camera Subsystem.
The detailed terms of the Working Group are provided in \citeds{LDM-702}.

\section{Conclusions}

\appendix

\section{References}
\label{sec:bib}
\bibliography{lsst,lsst-dm,refs_ads,refs,books}

\section{Abbreviations and Definitions}
\label{sec:acronyms}
\input{acronyms.tex}

\end{document}
