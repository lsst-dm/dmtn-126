\section{Aladinlite}

Aladinlite is a tool for viewing images on entire-sky scales that allows users
to zoom down to smaller scales defined at the time the images were ingested.
It enables this functionality by storing images in the HiPS (Hierarchical
Progressive Survey) schema [citation to 2015A\&A...578A.114F].  In the HiPS
schema, images are sampled onto iteratively more refined healpixel grids.  These
resamplings are stored on a central server which allows users connecting to that
server to smoothly transition between levels of refinement, down to the
original, finest healpixel sampling.  While this seems to enable much of the
functionality we have recommended, specifically the ability to transition from
full focal plane inspection down to single amplifier inspection, the fact that
the images must be resampled onto a pixel grid that is defined in sky
coordinates means that the raw, sensor-level pixel information is ultimately
lost.  Aladinlite was designed to enable scientists to explore images at the
entire-sky level.  It was not meant to enable engineers to inspect pixel data
from sensors.  While Aladinlite will likely have a role to play in serving LSST
image data to the scientific community, it is unlikely to be helpful during the
process of commissioning the LSST hardware and software.
